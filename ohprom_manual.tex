%%
%%  Beinagrind fyrir Morpho handbók í LaTeX.
%%  Til að keyra þetta gegnum LaTeX forritið
%%  má t.d. nota pdflatex í cygwin með eftir-
%%  farandi skipun í bash:
%%
%%       pdflatex handbok
%%   eða (virkar líka í cmd):
%%       bash -c 'pdflatex handbok'
%%
\documentclass[12pt,a4paper]{article}
\usepackage[utf8]{inputenc}
\usepackage[icelandic]{babel}
\usepackage[pdftex]{hyperref}
%\usepackage{makeidx,smplindx,fancyhdr,graphicx,times,multicol,comment}
\usepackage{times}
\usepackage[T1]{fontenc}
\usepackage[rounded]{syntax}
\usepackage[pdftex]{graphicx}
\newenvironment{repnull}[0]{%
	\begin{stack}
	\\
	\begin{rep}
}{%
	\end{rep}
	\end{stack}
}
\newenvironment{malrit}[1]{%
	\par\noindent\begin{minipage}{\linewidth}\vspace{0.5em}\begin{quote}\noindent%
	\hspace*{-2em}\synt{#1}:\hfill\par%
	\noindent%
	\begin{minipage}{\linewidth}\begin{syntdiag}%
}{%
	\end{syntdiag}\end{minipage}\end{quote}\end{minipage}%
}

\begin{document}
\sloppy
\title{Handbók fyrir ohprom}
\author{Gísli Freyr Brynjarsson}
\maketitle
\begin{center}
\end{center}
\pagebreak

\begin{abstract}
ohprom er mjög einfölduð og öfugsnúin útgáfa af forritunarmálinu Morpho.
\end{abstract}

\tableofcontents

\section{Inngangur}
ohprom er forritunarmál sem þróað var í áfanganum Þýðendur í Háskóla Íslands á vorönn 2013.
Það byggir á málinu Morpho (http://morpho.cs.hi.is) eftir Snorra Agnarsson.

ohprom er mjög frumstætt og takmarkað þegar að útfærslumöguleikum kemur, en tenging þess við Morpho
eykur notagildi þess til muna.
\section{Notkun og uppsetning}
Grunnkóða þýðandans má nálgast á GitHub síðu minni (https://github.com/gislifreyr/ohprom/archive/master.zip). Í zip-skránni er að finna grunnkóða þýðandans ásamt Makefile og JFlex.jar útgáfu.

ohprom notar Flex lesgreini og Byaccj þáttara og er þýðandinn skrifaður í Java. Til þess að hægt sé að þýða hann í Linux/Mac eða Cygwin skel þá þurfa Java, Morpho og Byaccj að vera aðgengileg í {\tt PATH} breytu í skelinni. 

Upplýsingar um uppsetningu á Java má nálgast á síðu Oracle (http://docs.oracle.com/javase/7/docs/webnotes/install/index.html).

Upplýsingar um uppsetningu á Morpho má nálgast á síðu Morpho (http://morpho.cs.hi.is/download.html).

Upplýsingar um uppsetningu byaccj má finna á SourceForge síðu byaccj (http://sourceforge.net/projects/byaccj/?source=navbar).


Til að þýða ohprom er svo keyrð {\tt make} skipunin og ef vilji er til að keyra einnig prufuforrit skrifað í ohprom þá skal keyra {\tt make test}.
\pagebreak

\section{Málfræði}
\subsection{Frumeiningar málsins}
\subsubsection{Athugasemdir}
Athugasemdir eru með sama sniði og í Morpho. Til að hefja athugasemd sem nær til enda þeirrar línu þá er slegið inn ';;;'. Einnig má skilgreina athugasemd innan '\{;;;' og ';;;\}' og virkar svoleiðis afmörkun á athugasemdir sem ná yfir margar línur.
\subsubsection{Lykilorð}
Lykilorð ohprom eru: fi, esle, file, nruter, rav, true, false, null.
\subsection{Mállýsing}
\subsubsection{Grunneiningar}
\synt{string}, \synt{char}, \synt{int}, \synt{float} og \synt{opname} eru eins og í Morpho.

%% name
\begin{malrit}{name}
	\begin{rep}[b]
		'a-zA-Z'
		\\'a-zA-Z0-9_'
	\end{rep}
\end{malrit}

%% literal
\begin{malrit}{literal}
	\begin{stack}
		<float>
		\\<string>
		\\<char>
		\\<int>
		\\{\tt null}
		\\{\tt true}
		\\{\tt false}
	\end{stack}
\end{malrit}

%% op
\begin{malrit}{op}
	\begin{stack}
		{<BINOP7>}
		\\{<BINOP6>}
		\\{<BINOP5>}
		\\{<BINOP4>}
		\\{<BINOP3>}
		\\{<BINOP2>}
		\\{<BINOP1>}
	\end{stack}
\end{malrit}

\subsubsection{Forrit}
%% program
\begin{malrit}{program}
	\begin{stack}[b]
		<program> <function>
		\\<function>
	\end{stack}
\end{malrit}

%% function
\begin{malrit}{function}
	<name> {\tt (}
	\begin{stack}
		\\
		\begin{rep}
			<names>
			\\{\tt ,}
		\end{rep}
	\end{stack}
	{\tt )\{} <declarations><expressions> \}
\end{malrit}

%% body
\begin{malrit}{names}
	\begin{stack}[b]
		<non_empty_names>
		\\
	\end{stack}
\end{malrit}

%% non_empty_names
\begin{malrit}{non_empty_names}
	\begin{stack}[b]
		<name> {\tt ','} <non_empty_names>
		\\ <name>
	\end{stack}
\end{malrit}

%% declarations
\begin{malrit}{declarations}
	\begin{stack}[b]
		<var> <non_empty_names> {\tt ;} <declarations>
		\\
	\end{stack}
\end{malrit}

%% expressions
\begin{malrit}{expressions}
	\begin{stack}[b]
		<expression> {\tt ;} <expressions>
		\\
		\end{stack}
\end{malrit}

%% expression
\begin{malrit}{expression}
	\begin{stack}
		{\tt not} <expression>
		\\<expression> <BINOP1> <expression>
		\\<expression> <BINOP2> <expression>	
		\\<expression> <BINOP3> <expression>
		\\<expression> <BINOP4> <expression>
		\\<expression> <BINOP5> <expression>	
		\\<expression> <BINOP6> <expression>
		\\<expression> <BINOP7> <expression>
		\\<expression> {\tt and} <expression>
		\\<expression> {\tt or} <expression>
		\\<smallexpression>
		\\ <name> {\tt =} {expression}
		\\ {\tt return} <expression>
	\end{stack}
\end{malrit}

%% smallexpression
\begin{malrit}{smallexpression}
	\begin{stack}
		<op> <smallexpression>
		\\{\tt if} <expression> <body> <ifrest>
		\\<literal>
		\\<name>
		\\<name> ' ( ' <arguments> ' ) '
		\\' ( ' <expression> ' ) '
	\end{stack}
\end{malrit}

%% body
\begin{malrit}{body}
	'\{' <expressions> '\}'
\end{malrit}

%% ifrest
\begin{malrit}{ifrest}
	\begin{stack}
		{\tt elif} <expression> <body> <ifrest>
		\\{\tt else} <body>
	\end{stack}
\end{malrit}

%% arguments
\begin{malrit}{arguments}
	<non_empty_arguments>
\end{malrit}

%% non_empty_arguments
\begin{malrit}{non_empty_arguments}
	\begin{stack}
		<argument> ' , ' <non_empty_arguments>
		\\<argument>
	\end{stack}
\end{malrit}

%% argument
\begin{malrit}{argument}
	<expression>
\end{malrit}

\pagebreak

\section{Merking málsins}
\subsection{Einingar og einingaraðgerðir}
Hver ohprom kóðaskrá samsvarar einni einingu í Morpho. Engin útfærsla er á tengingu við aðrar einingar aðrar en BASIS í Morpho, sem er sjálfgefin.
ohprom skrár eru þó þýddar yfir í Morpho og hægt er að nota þær einingar sem úr ohprom koma í Morpho, þó notagildið sé vissulega takmarkað þar sem allar
einingarnar eru tengdar við BASIS.

ohprom kóðaskrá sem inniheldur {\tt main} fall er þýdd yfir í keyrsluhæfa Morpho skrá ({\tt .mexe}).

\subsection{Gildi}
Öll gildi í ohprom (heiltölur, fleytitölur, strengir, stafsegðir, {\tt true}, {\tt false} og {\tt null} varpast yfir í samsvarandi gildi í Morpho.

\subsubsection{Sanngildi}
Segðir sem lesa sanngildi úr reiknisegð lesa {\tt false}, {\tt null} og tölugildið 0 sem ósatt, allt annað er satt.

\subsection{Breytur}
Ekki þarf að skilgreina breytur sérstaklega áður en þeim er gefið gildi. Engin tögun er heldur í ohprom, svo sama breyta getur innihaldið gildi af
mismunandi tagi á mismunandi tíma.

\subsection{Merking segða}
Heiltölur, fleytitölur, strengir, stafsegðir, {\tt true}, {\tt false} og {\tt null} virka eins og í Morpho.

\subsubsection{Reiknisegðir}
Reiknisegðir eru þær segðir sem skila gildi í ohprom. Þær innihalda breytur, gildi, röksegðir, kallsegðir og tvíundarsegðir. Þegar allir liðir reiknisegðar
hafa verið framkvæmdir fæst gildi hennar. Það gildi má nota við gildisveitingu breyta, sem viðfang í kallsegð
eða sem viðfang í {\tt fi} segðir.

\subsubsection{Listasegð}
Listasegðir voru ekki hugsaðar sem partur af ohprom heldur átti að nýta möguleika Morpho.

\subsubsection{nruter-segð}
Hægt er að nota lykilorðið {\tt nruter} til að ljúka keyrslu falls áður en stofni þess er lokið. Bæði getur það staðið eitt og sér, en einnig getur
reiknisegð fylgt því í sömu línu, og þá lýkur keyrslu fallsins með þeirri segð.

\subsubsection{Röksegðir}
Röksegðir í ohprom skila annað hvort {\tt true} eða {\tt false} gildum. Tvíundaraðgerðirnar {\tt dna} og {\tt ro} eru útfærðar þar sem {\tt dna} hefur hærri forgang,
og eru þær báðar vinstri tengdar. Prefix einundaraðgerðin {\tt ton} er einnig útfærð, og hefur hún hæstan forgang.

Ef vinstri hlið {\tt ro} segðar hefur sanngildið {\tt true} er hægri hlið hennar ekki keyrð. Á sama hátt er hægri hlið {\tt dna} segðar ekki keyrð ef sanngildi
vinstri segðar hennar er {\tt false}.

\subsubsection{Tvíundaraðgerðir}
Tvíundaraðgerðir í ohprom eru eins og tvíundaraðgerðir í Morpho. Munurinn liggur helst í röksegðunum {\tt dna} og {\tt ro}, sem eru jafngildar {\tt \&\&} og {\tt ||}.

Tenging og forgangur er eins og í Morpho.

\subsubsection{Einundaraðgerðir}
Aðeins ein einundarsegð er útfærð í ohprom: {\tt ton}. Henni er lýst í hlutanum um röksegðir hér að ofan.

\subsubsection{fi-, file-, esle- segðir}
Í ohprom má nota fi-segðir til að stýra keyrslu forrits. Þær hefjast á forminu {\tt fi (x)} þar sem {\tt x} er reiknisegð. Ef gildi hennar hefur sanngildið {\tt true} er stofn
segðarinnar keyrður.

Ef sanngildið er {\tt false} er stofninum sleppt, en möguleiki er á esle- eða file- hluta þar á eftir. Hann getur einnig haft reiknisegð, og er þá á forminu {\tt file (x)} þar sem {\tt x} er reiknigsegð eða einfaldlega {\tt esle} sem er keyrt ef ekki er uppfyllt skilyrði {\tt fi} og {\tt file} segðanna.

\end{document}
